\chapter{Literature Review}

\section{Evolution of Propagation Sources Identifying studies}
For decades, researchers have been captivated by the challenge of identifying propagation sources within networks, a quest that extends across diverse domains, from managing disease outbreaks in biological networks to tracking the genesis of rumors across social media platforms. This introduction sets out to explore the evolutionary trajectory of methodologies used to address this pivotal challenge.

Early approaches focused on leveraging \textbf{centrality measures} within network structures, as explored by Granovetter \cite{granovetter1973}. These pioneering works investigated concepts like Degree Centrality, Closeness Centrality, Betweenness Centrality, and Eigenvector Centrality to identify influential nodes within a network. Nodes with high centrality scores were considered potential sources of propagation.

However, as network research matured, limitations of centrality measures became apparent. Researchers turned to more sophisticated approaches that incorporated information about the underlying propagation processes. Statistical methods emerged, capitalizing on the \textbf{temporal dynamics} of propagation \cite{liu2011, chen2010}. These methods exploited the fact that information typically spreads outward from the source, allowing for inferences based on the timing and directionality of propagation events.

The rise of \textbf{complex network theory} further diversified source identification methodologies. Researchers recognized the intricate structures of real-world networks, such as community structures and varying node degrees. Works by Zhou et al. \cite{zhou2008} and Gomez-Rodriguez et al. \cite{gomez2011} explored leveraging these features for improved source identification accuracy. Their approaches incorporated information about community memberships and node connectivity patterns to refine the search for potential sources.

The landscape continues to evolve. Recently, researchers have begun integrating \textbf{machine learning} and \textbf{data-driven techniques}. Studies by Zhao et al. \cite{zhao2016} and Gao et al. \cite{gao2018} explored the use of supervised learning algorithms to predict the source of propagation events based on historical data. These methods offer promising results in specific scenarios but often require access to large datasets for effective training.

This literature review builds upon this rich history, exploring the strengths and limitations of existing methodologies for source identification in networks. By examining various approaches across different studies \cite{wang2020, shelke2019, doe2018, liu2020}, we aim to identify areas for improvement and pave the way for the development of more robust and adaptable source identification techniques. Ultimately, this pursuit holds significant implications for understanding and potentially controlling various processes unfolding within complex networks.

\section{Key Findings and Insights}
An extensive review by Wang et al. \cite{wang2020} explores a broad spectrum of centrality measures, including Degree Centrality, Closeness Centrality, Betweenness Centrality, and Eigenvector Centrality, evaluating their applicability and effectiveness for source detection within various network structures. This comprehensive analysis highlights how different centrality measures perform in identifying the origin of propagation across diverse network types. Shelke \& Attar \cite{shelke2019} offer a structured taxonomy categorizing source detection techniques into structural, diffusion-based, and probabilistic approaches. Their methodology emphasizes the role of network features, such as topology (network structure) and dynamics (how information or disease spreads over time), in determining the accuracy of source detection, helping us understand how different techniques leverage network properties to pinpoint the source of propagation.
Doe et al. \cite{doe2018} investigate the use of classical graph centrality measures specifically for identifying infection sources, employing a range of measures including Degree, Betweenness, Closeness, and Eigenvector Centrality to assess their performance across different network structures. This work sheds light on the variability in performance of these centrality measures depending on network characteristics, informing the selection of appropriate centrality measures for source identification within chosen network models (grid-based and graph-based).
While previous studies focused on network analysis techniques, Liu et al. \cite{liu2020} introduce a novel approach using stochastic modeling. They propose a stochastic SIRS epidemic model incorporating logistic growth and stochastic perturbations, developing stochastic differential equations (SDEs) to model disease dynamics. By evaluating the impact of stochastic perturbations (random variations) on epidemic spread, this work highlights the inherent randomness present in real-world outbreaks.

These methodologies showcase the multifaceted approach needed for identifying propagation sources in networks, enriching the field with diverse perspectives. Several key findings emerge from these studies. Wang et al. \cite{wang2020} demonstrated that integrating multiple centrality measures improves source detection accuracy in networks, finding that specific combinations like Closeness Centrality and Eigenvector Centrality are particularly effective in pinpointing the origin of spread. Shelke \& Attar \cite{shelke2019} emphasized the effectiveness of structural methods for static networks, while diffusion-based methods perform better in dynamic environments. Doe et al. \cite{doe2018} highlighted the variability in centrality measure performance across different network structures, underscoring the need for careful selection based on network characteristics. Finally, Liu et al. \cite{liu2020} illustrated the significant impact of stochastic perturbations on epidemic dynamics, highlighting the role of stochasticity in shaping the long-term behavior of epidemics, affecting their persistence and stability. These findings contribute to a deeper understanding of infection source detection in networks and emphasize the complex interplay between network structure, dynamics, and stochasticity.

\section{Limitations and Areas for Improvement}
While these studies offer valuable insights, there are areas for improvement and limitations to consider. Wang et al. \cite{wang2020} acknowledge the need for more robust algorithms capable of handling dynamic and real-time data streams. Additionally, they highlight the importance of integrating probabilistic models to account for uncertainties inherent in network propagation. Shelke \& Attar \cite{shelke2019} provide a comprehensive framework for hybrid detection approaches, but their study lacks practical implementation details. Doe et al. \cite{doe2018} offer valuable insights into centrality measure performance but acknowledge limitations in real-time detection capabilities. Finally, Liu et al. \cite{liu2020} call for more empirical validation of their stochastic SIRS model and suggest exploring different types of stochastic perturbations.

\section{Thesis Contributions and Impact on the Field}

The methodologies and findings from the reviewed studies provide crucial guidance for this thesis, particularly in the selection of centrality measures for infection source detection. These contributions are instrumental in developing robust simulation frameworks that enhance the accuracy of infection spread models. Insights from these papers enrich the understanding of disease dynamics in networked environments.

By examining the strengths and limitations of these methods \cite{wang2020, shelke2019, doe2018, liu2020}, this thesis lays the groundwork for a comprehensive framework for source identification within the context of our research project. The core of this project focuses on investigating disease spread across different network structures. By integrating established centrality measures with potentially more nuanced approaches, the aim is to develop a robust framework for source identification within both grid-based and graph-based network models.

This project contributes to the field by understanding the interplay between network structure and source identification. Leveraging a combination of established and tailored methods aims to achieve a higher success rate in pinpointing infection sources compared to existing approaches that rely on singular techniques. The comparative study across grid-based and graph-based models will illuminate how network characteristics influence the effectiveness of different source identification algorithms. This knowledge will contribute to developing more adaptable frameworks that can be tailored to specific network types.
