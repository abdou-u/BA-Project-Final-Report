\chapter{Literature Review}

\section{A Historical Perspective on Identifying Propagation Sources in Networks}
For decades, researchers have been captivated by the challenge of identifying propagation sources within networks, a quest that extends across diverse domains, from managing disease outbreaks in biological networks to tracking the genesis of rumors across social media platforms. This introduction sets out to explore the evolutionary trajectory of methodologies used to address this pivotal challenge.\\

Early approaches focused on leveraging \textbf{centrality measures} within network structures, as explored by Granovetter \cite{granovetter1973}. These pioneering works investigated concepts like Degree Centrality, Closeness Centrality, Betweenness Centrality, and Eigenvector Centrality to identify influential nodes within a network. Nodes with high centrality scores were considered potential sources of propagation.\\

However, as network research matured, limitations of centrality measures became apparent. Researchers turned to more sophisticated approaches that incorporated information about the underlying propagation processes. Statistical methods emerged, capitalizing on the \textbf{temporal dynamics} of propagation \cite{liu2011, chen2010}. These methods exploited the fact that information typically spreads outward from the source, allowing for inferences based on the timing and directionality of propagation events.\\

The rise of \textbf{complex network theory} further diversified source identification methodologies. Researchers recognized the intricate structures of real-world networks, such as community structures and varying node degrees. Works by Zhou et al. \cite{zhou2008} and Gomez-Rodriguez et al. \cite{gomez2011} explored leveraging these features for improved source identification accuracy. Their approaches incorporated information about community memberships and node connectivity patterns to refine the search for potential sources.\\

The landscape continues to evolve. Recently, researchers have begun integrating \textbf{machine learning} and \textbf{data-driven techniques}. Studies by Zhao et al. \cite{zhao2016} and Gao et al. \cite{gao2018} explored the use of supervised learning algorithms to predict the source of propagation events based on historical data. These methods offer promising results in specific scenarios but often require access to large datasets for effective training.\\

This literature review builds upon this rich history, exploring the strengths and limitations of existing methodologies for source identification in networks. By examining various approaches across different studies \cite{wang2020, shelke2019, doe2018, liu2020}, we aim to identify areas for improvement and pave the way for the development of more robust and adaptable source identification techniques. Ultimately, this pursuit holds significant implications for understanding and potentially controlling various processes unfolding within complex networks.

\section{Methodologies in Propagation Source Identification}
\subsection{Centrality Measures in Source Detection}
This extensive review by Wang et al. \cite{wang2020} explores a broad spectrum of centrality measures, including Degree Centrality, Closeness Centrality, Betweenness Centrality, and Eigenvector Centrality. They evaluate the applicability and effectiveness of these measures for source detection within various network structures. Their work provides a comprehensive understanding of how different centrality measures perform in identifying the origin of propagation across diverse network types.

\subsection{Taxonomy of Source Detection Techniques}
Shelke \& Attar \cite{shelke2019} offer a structured taxonomy that categorizes source detection techniques into three main categories: structural, diffusion-based, and probabilistic approaches. Their methodology emphasizes the role of network features such as topology (network structure) and dynamics (how information or disease spreads over time) in determining the accuracy of source detection. This categorization helps us understand how different techniques leverage network properties to pinpoint the source of propagation.

\subsection{Centrality Measures for Infection Source Identification}
Doe et al. \cite{doe2018} investigate the use of classical graph centrality measures specifically for identifying infection sources. They employ a range of measures, including Degree, Betweenness, Closeness, and Eigenvector Centrality, to assess their performance across different network structures. Their work sheds light on the variability in performance of these centrality measures depending on network characteristics. This knowledge informs our selection of appropriate centrality measures for source identification within our chosen network models (grid-based and graph-based).

\subsection{Stochastic Modeling of Epidemic Spread}
While the previous studies focused on network analysis techniques, Liu et al. \cite{liu2020} introduce a novel approach using stochastic modeling. They propose a stochastic SIRS epidemic model that incorporates logistic growth and stochastic perturbations. Their methodology involves developing stochastic differential equations (SDEs) to model disease dynamics. By evaluating the impact of stochastic perturbations (random variations) on epidemic spread, this work highlights the inherent randomness present in real-world outbreaks.\\

The methodologies employed in these studies showcase the multifaceted approach needed for identifying propagation sources in networks, enriching the field with diverse perspectives.

\section{Key Findings and Insights}
Across these studies, several key findings emerge. Wang et al. \cite{wang2020} demonstrated that integrating multiple centrality measures improves source detection accuracy in networks. They found that specific combinations like Closeness Centrality and Eigenvector Centrality are particularly effective in pinpointing the origin of spread. Shelke \& Attar \cite{shelke2019} emphasized the effectiveness of structural methods for static networks, while diffusion-based methods perform better in dynamic environments. Doe et al. \cite{doe2018} highlighted the variability in centrality measure performance across different network structures, underscoring the need for careful selection based on network characteristics. Finally, Liu et al. \cite{liu2020} illustrated the significant impact of stochastic perturbations on epidemic dynamics. Their work highlighted the role of stochasticity in shaping the long-term behavior of epidemics, affecting their persistence and stability. These findings contribute to a deeper understanding of infection source detection in networks and emphasize the complex interplay between network structure, dynamics, and stochasticity.

\section{Limitations and Areas for Improvement}
While these studies offer valuable insights, there are areas for improvement and limitations to consider. Wang et al. \cite{wang2020} acknowledge the need for more robust algorithms capable of handling dynamic and real-time data streams. Additionally, they highlight the importance of integrating probabilistic models to account for uncertainties inherent in network propagation. Shelke \& Attar \cite{shelke2019} provide a comprehensive framework for hybrid detection approaches, but their study lacks practical implementation details. Doe et al. \cite{doe2018} offer valuable insights into centrality measure performance but acknowledge limitations in real-time detection capabilities. Finally, Liu et al. \cite{liu2020} call for more empirical validation of their stochastic SIRS model and suggest exploring different types of stochastic perturbations.\\

Despite these limitations, these studies collectively advance the understanding of infection source detection in networks. They offer valuable insights into the selection of centrality measures and hybrid approaches, contribute to the development of simulation frameworks, and enhance the accuracy of infection spread models. Moreover, they highlight the importance of considering network structure, dynamics, and stochasticity when understanding disease spread in networks \cite{wang2020, shelke2019, doe2018, liu2020}.

\section{Use for the Thesis and Main Contribution to the Field}
\subsection{Use for the Thesis}
The methodologies and findings from these studies directly inform the thesis, providing guidance on the selection of centrality measures and hybrid approaches for infection source detection. They contribute to the development of simulation frameworks and enhance the accuracy of infection spread models. The insights gleaned from these papers and books enrich the understanding of disease dynamics in networked environments, facilitating the design of more effective containment measures and intervention strategies. Additionally, they offer valuable perspectives on the complex interplay between network structure, dynamics, and stochasticity in shaping disease spread, providing a solid foundation for further research in the field.\\

By examining the strengths and limitations of these methods \cite{wang2020, shelke2019, doe2018, liu2020}, we aim to lay the groundwork for a comprehensive framework for source identification within the context of our research project.

\subsection{Main Contribution to the Field}
The insights gleaned from this review inform the core of our project – investigating disease spread across different network structures. By integrating established centrality measures with potentially more nuanced approaches, we aim to develop a robust framework for source identification within both grid-based and graph-based network models. This project contributes to the field by:
\begin{itemize}
    \item \textbf{Enhancing source identification accuracy}: By leveraging a combination of established and potentially more tailored methods, we aim to achieve a higher success rate in pinpointing infection sources compared to existing approaches that rely on singular techniques.
    \item \textbf{Understanding the interplay between network structure and source identification}: Our comparative study across grid-based and graph-based models will illuminate how network characteristics influence the effectiveness of different source identification algorithms. This knowledge will contribute to the development of more adaptable frameworks that can be tailored to specific network types.
    \item \textbf{Informing disease containment strategies}: Improved source identification accuracy translates to faster intervention and containment measures. By providing a more comprehensive understanding of source detection within different network structures, this project has the potential to inform the development of more effective disease control strategies.
\end{itemize}

\section{Conclusion}
The reviewed studies showcase the multifaceted approach needed for identifying propagation sources in networks. They offer valuable insights into the selection of centrality measures, hybrid approaches, and the development of simulation frameworks to enhance the accuracy of infection spread models.