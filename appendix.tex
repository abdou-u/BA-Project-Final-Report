\appendix
\chapter{Algorithm Implementations}

\section{Jordan Center}

\begin{lstlisting}[caption=Jordan Centrality using BFS, label=lst:jordan-center]
import numpy as np
import time
from collections import deque

def bfs(grid, start_pos):
    # Implement BFS to calculate distances
    pass  # Omitted for brevity

def infection_nodes(grid):
    # Identify all infected or previously infected nodes
    pass

def find_infection_source_bfs(grid):
    height, width = grid.shape

    # For SIRS, we take nodes that are infected or were infected
    candidate_nodes = infection_nodes(grid)

    centrality_scores = np.zeros_like(grid, dtype=np.float32)

    global_start_time = time.time()
    for i, start_pos in enumerate(candidate_nodes):
        start_time = time.time()
        start_pos_tuple = tuple(start_pos)
        
        # Pass the precomputed infected nodes if applicable
        distances = bfs(grid, start_pos_tuple)

        # Compute centrality score considering distances to other nodes
        reachable_nodes = distances[distances != -1]  # Exclude unreachable nodes
        num_reachable_nodes = len(reachable_nodes) - 1  # Exclude the start node itself
        total_distance = np.sum(reachable_nodes)
        
        if num_reachable_nodes > 0:
            average_distance = total_distance / num_reachable_nodes
            centrality_scores[start_pos_tuple] = 1 / average_distance
        else:
            centrality_scores[start_pos_tuple] = 0

    # Find the node with the highest centrality score
    source_node = np.unravel_index(np.argmax(centrality_scores), grid.shape)
    
    return source_node
\end{lstlisting}

\section{Center of Mass}

\begin{lstlisting}[caption=Center of Mass Algorithm, label=lst:center-of-mass]
import numpy as np

def flood_fill(grid, mask, start_i, start_j, target_val, fill_value):
    stack = {(start_i, start_j)}  # Using a set to avoid duplicate positions
    while stack:
        x, y = stack.pop()
        if mask[x, y] != fill_value and grid[x, y] == target_val:
            mask[x, y] = fill_value
            base_idx = (x * width + y) * 4
            for k in range(4):  # Changed loop variable to k to avoid confusion with i, j
                neighbor_idx = neighbors_map[base_idx + k]
                if neighbor_idx != -1:  # Valid neighbor index
                    nx, ny = divmod(neighbor_idx, width)  # Convert flat index to 2D indices
                    if mask[nx, ny] != fill_value:
                        stack.add((nx, ny))

def find_external_boundary_grid(grid):
    height, width = grid.shape
    infected_val = 0  # Infected nodes
    susceptible_val = -1  # Susceptible nodes

    # Step 1: Identify All Potential Boundary Nodes
    potential_boundary = np.zeros_like(grid, dtype=bool)
    for i in range(height):
        for j in range(width):
            if grid[i, j] == infected_val:
                base_idx = (i * width + j) * 4
                for k in range(4):
                    neighbor_idx = neighbors_map[base_idx + k]
                    if neighbor_idx != -1:
                        nx, ny = divmod(neighbor_idx, width)
                        if grid[nx, ny] == susceptible_val:
                            potential_boundary[i, j] = True
                            break

    # Step 2: Flood Fill from Grid Edges
    mask = np.zeros_like(grid, dtype=bool)
    edge_indices = np.hstack((np.arange(0, width), np.arange(grid.size - width, grid.size)))
    for edge_idx in edge_indices:
        i, j = divmod(edge_idx, width)
        if grid[i, j] == susceptible_val and not mask[i, j]:
            flood_fill(grid, mask, i, j, susceptible_val, True)

    # Step 3: Identify Internal Potential Boundary Nodes
    internal_potential_boundary = np.zeros_like(grid, dtype=bool)
    for i in range(height):
        for j in range(width):
            if potential_boundary[i, j]:
                base_idx = (i * width + j) * 4
                for k in range(4):
                    neighbor_idx = neighbors_map[base_idx + k]
                    if neighbor_idx != -1:
                        nx, ny = divmod(neighbor_idx, width)  # Convert flat index to 2D indices
                        if mask[nx, ny]:
                            internal_potential_boundary[i, j] = True
                            break

    # Step 4: Isolate Actual Boundary Nodes
    actual_boundary = potential_boundary & internal_potential_boundary

    return actual_boundary

def find_centroid_mean(external_boundary_grid):
    boundary_indices = np.argwhere(external_boundary_grid)
    
    # Calculate the centroid (mean position) of the boundary points.
    centroid_x, centroid_y = np.mean(boundary_indices, axis=0)
    
    # The centroid coordinates are given as (centroid_x, centroid_y).
    # If you need to round them to the nearest integer (since grid indices are integers), you can do:
    centroid_x = int(round(centroid_x))
    centroid_y = int(round(centroid_y))
    
    centroid = (centroid_x, centroid_y)
    return centroid
    
centroid = find_centroid_mean(find_external_boundary_grid(grid))
print(centroid)
\end{lstlisting}

\section{Distance Analysis}

\begin{lstlisting}[caption=Distance Analysis Algorithm, label=lst:distance-analysis]
import numpy as np
from numba import jit
from collections import deque

@jit(nopython=True)
def max_euc_distance_to_boundary(x0, y0, boundary_indices):
    """Calculate the maximum Euclidean distance from (x0, y0) to any boundary point."""
    max_distance = 0.0
    for index in range(boundary_indices.shape[0]):
        x, y = boundary_indices[index]
        euc_distance = np.sqrt((x - x0) ** 2 + (y - y0) ** 2)
        if euc_distance > max_distance:
            max_distance = euc_distance
    return max_distance

def find_infection_source_euc_distance(grid):
    """Find the source of infection by minimizing the maximum distance to the external boundary."""
    height, width = grid.shape
    start_pos = (height // 2, width // 2)

    # Convert the external boundary to a NumPy array of indices for efficient processing
    external_boundary_grid = find_external_boundary_grid(grid)
    external_boundary_indices = np.argwhere(external_boundary_grid)
    
    # Initialize queue and visited set
    queue = deque([start_pos])
    visited = set([start_pos])
    current_best_node = start_pos
    current_min_max_distance = max_euc_distance_to_boundary(start_pos[0], start_pos[1], external_boundary_indices)

    while queue:
        x0, y0 = queue.popleft()

        # Check all neighbors to find the one with the smallest maximum distance to the boundary
        min_neighbor_distance = float('inf')
        next_node = None

        base_index = (x0 * width + y0) * 4
        for i in range(4):
            neighbor_idx = neighbors_map[base_index + i]
            if neighbor_idx == -1:
                continue

            nx0, ny0 = divmod(neighbor_idx, width)
            if (nx0, ny0) not in visited:
                visited.add((nx0, ny0))
                n_max_distance = max_euc_distance_to_boundary(nx0, ny0, external_boundary_indices)
                if n_max_distance < min_neighbor_distance:
                    min_neighbor_distance = n_max_distance
                    next_node = (nx0, ny0)
                    
        if next_node and min_neighbor_distance < current_min_max_distance:
            # Only proceed if the next node improves the maximum distance
            queue.append(next_node)
            current_min_max_distance = min_neighbor_distance
            current_best_node = next_node

    return current_best_node
\end{lstlisting}

\section{Gillespie Algorithm}

\begin{lstlisting}[caption=Gillespie Algorithm for SIRS Model, label=lst:gillespie-algorithm]
import numpy as np

def gillespie_sirs(S0, I0, R0, beta, gamma, delta, T):
    S, I, R = [S0], [I0], [R0]
    t = 0

    while t < T:
        N = S[-1] + I[-1] + R[-1]
        rates = [
            beta * S[-1] * I[-1] / N,  # Infection rate
            gamma * I[-1],             # Recovery rate
            delta * R[-1]              # Loss of immunity rate
        ]
        rate_sum = sum(rates)

        if rate_sum == 0:
            break

        tau = np.random.exponential(1 / rate_sum)
        t += tau
        rand = np.random.random() * rate_sum

        if rand < rates[0]:
            S.append(S[-1] - 1)
            I.append(I[-1] + 1)
            R.append(R[-1])
        elif rand < rates[0] + rates[1]:
            S.append(S[-1])
            I.append(I[-1] - 1)
            R.append(R[-1] + 1)
        else:
            S.append(S[-1] + 1)
            I.append(I[-1])
            R.append(R[-1] - 1)

    return np.array(S), np.array(I), np.array(R)
\end{lstlisting}