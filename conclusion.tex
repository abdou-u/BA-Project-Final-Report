\chapter{Conclusion and Future Scope}

In conclusion, this thesis has provided a detailed examination of infectious disease spread across various network structures. The study explored both grid-based algorithms, which are fundamental in epidemiological studies due to their simplicity and structured nature, and more complex network models focusing on understanding disease spread dynamics.  The aim is to improve the traceability of infection sources, which are crucial for a comprehensive understanding of disease spread dynamics in real-world scenarios. \\ 

The research highlights the importance of selecting appropriate modeling approaches to accurately simulate and trace infection dynamics.  By integrating grid-based, network-based, and stochastic models, this work offers valuable insights into the complexities of disease transmission and the potential for improved disease monitoring and control.\\

The proposed future research directions aim to build on these findings, enhancing the robustness and applicability of simulation models in real-world contexts. Continued investigation in these areas will contribute to the ongoing efforts to understand and combat infectious diseases, ultimately supporting public health initiatives and improving outcomes for populations worldwide.