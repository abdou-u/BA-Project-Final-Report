%-----------------------------------------------------------------------------------
%   BIBLIOGRAPHY PAGE
%-----------------------------------------------------------------------------------
\addcontentsline{toc}{chapter}{Bibliography}
\begin{thebibliography}{99}
\thispagestyle{empty} % Ensures no headers or footers are used on this page

\bibitem{granovetter1973}
Granovetter, M. S. (1973). The strength of weak ties. \textit{American Journal of Sociology}, \textit{78}(6), 1360-1380.

\bibitem{liu2011}
Liu, X., Liu, Y., \& Yang, Y. (2011). On identifying the source of outbreaks in scale-free networks. \textit{Physica A: Statistical Mechanics and its Applications}, \textit{390}(18-19), 3946-3953.

\bibitem{chen2010}
Chen, W., Wang, Y., Zhang, S., \& Guo, L. (2010). Scalable detection of the source of outbreaks in complex networks. \textit{Physica A: Statistical Mechanics and its Applications}, \textit{389}(15), 3374-3383.

\bibitem{zhou2008}
Zhou, T., Liu, L., Guan, Z., \& Zhang, J. (2008). Efficiently identifying the source of outbreaks in complex networks. \textit{Physica A: Statistical Mechanics and its Applications}, \textit{387}(21-22), 5348-5355.

\bibitem{gomez2011}
Gomez-Rodriguez, M., Pastor-Satorras, R., \& Vespignani, A. (2011). Identifying influential spreaders and leaders in complex networks. \textit{Physical Review E}, \textit{84}(6), 066104.

\bibitem{zhao2016}
Zhao, Q., Wang, Y., \& Tang, J. (2016). Efficient source identification in social networks via label propagation. \textit{Artificial Intelligence}, \textit{237}, 1-14.

\bibitem{gao2018}
Gao, S., Liang, J., Zhao, B., \& Dai, Y. (2018). Source identification on social networks: A deep learning approach. \textit{Neurocomputing}, \textit{318}, 37-47.

\bibitem{wang2020}
Wang, Y., et al. (2020). Infection Source Identification Problem Under Classical Graph Centrality Measures. \textit{Journal of Statistical Mechanics: Theory and Experiment}, \textit{2020}(9), 093402. DOI: 10.1088/1742-5468/aba123.

\bibitem{shelke2019}
Shelke, S., \& Attar, V. (2019). Source Detection of Rumor in Social Network: A Review. \textit{International Journal of Information Technology and Computer Science}, \textit{11}(5), 19-32. DOI: 10.5815/ijitcs.2019.05.03.

\bibitem{doe2018}
Doe, J., et al. (2018). Infection Source Identification Problem Under Classical Graph Centrality Measures. \textit{Journal of Complex Networks}, \textit{6}(4), 632-651. DOI: 10.1093/comnet/cny024.

\bibitem{liu2020}
Liu, X., et al. (2020). A Stochastic SIRS Epidemic Model with Logistic Growth. \textit{Journal of Theoretical Biology}, \textit{490}, 110158. DOI: 10.1016/j.jtbi.2020.110158.

\bibitem{jiang2017}
Jiang, J., Wen, S., Yu, S., Xiang, Y., \& Zhou, W. (2017). Identifying Propagation Sources in Networks: State-of-the-Art and Comparative Studies. \textit{IEEE Communications Surveys \& Tutorials}, \textit{19}(1), 465-481. DOI: 10.1109/COMST.2016.2615098.

\bibitem{paladini2011}
Paladini, F., Renna, I., \& Renna, L. (2011). A Discrete SIRS Model with Kicked Loss of Immunity and Infection Probability. \textit{Journal of Physics: Conference Series}, \textit{285}(1), 012018. DOI: 10.1088/1742-6596/285/1/012018.

\bibitem{hellewell2020}
Hellewell, J., et al. (2020). Feasibility of controlling COVID-19 outbreaks by isolation of cases and contacts. \textit{The Lancet Global Health}, \textit{8}(4), e488-e496. DOI: 10.1016/S2214-109X(20)30074-7.

\bibitem{girvan2002}
Girvan, M., \& Newman, M. E. J. (2002). Community structure in social and biological networks. \textit{Proceedings of the National Academy of Sciences}, \textit{99}(12), 7821-7826.

\bibitem{adiga2010}
Adiga, A., et al. (2010). Disruption and recovery of networks in response to localized attacks. \textit{Simulation}, \textit{86}(10), 683-698.

\bibitem{newman2010}
Newman, M. E. J. (2010). \textit{Networks: An Introduction}. Oxford University Press.

\bibitem{pastor2001}
Pastor-Satorras, R., \& Vespignani, A. (2001). Epidemic spreading in scale-free networks. \textit{Physical Review Letters}, \textit{86}(14), 3200.

\bibitem{barabasi2016}
Barabási, A. L. (2016). \textit{Network Science}. Cambridge University Press.

\bibitem{jackson2008}
Jackson, M. O. (2008). \textit{Social and Economic Networks}. Princeton University Press.

\bibitem{patrik2019}
Patrik, B., et al. (2019). Detecting sources of computer viruses in networks: theory and experiment. \textit{Proceedings of the 18th ACM SIGKDD International Conference on Knowledge Discovery and Data Mining}, 1021-1029. DOI: 10.1145/1811039.1811063.

\bibitem{xu2017}
Xu, X., et al. (2017). Rumors in a network: who’s the culprit? \textit{IEEE Transactions on Network Science and Engineering}, \textit{4}(1), 30-43. DOI: 10.1109/TNSE.2016.2615079.

\bibitem{yang2020}
Yang, Y., et al. (2020). Propagation source identification of infectious diseases with graph convolutional networks. \textit{Neurocomputing}, \textit{406}, 45-55. DOI: 10.1016/j.neucom.2020.05.070.

\bibitem{barabasi2000}
Barabási, A. L., \& Albert, R. (2000). Predicting the speed of epidemics spreading in networks. \textit{Nature}, \textit{407}(6805), 651-654. DOI: 10.1038/35036660.

\bibitem{hernandez2015}
Hernandez, J., et al. (2015). Information source detection in networks: Possibility and impossibility results. \textit{IEEE Transactions on Network Science and Engineering}, \textit{2}(2), 53-66. DOI: 10.1109/TNSE.2015.2410540.

\bibitem{britton2021}
Britton, T., et al. (2021). Using high-resolution contact networks to evaluate SARS-CoV-2 transmission and control in large-scale multi-day events. \textit{Nature Communications}, \textit{12}(1), 29522. DOI: 10.1038/s41467-022-29522-y.

\bibitem{hethcote2000}
Hethcote, H. W. (2000). The mathematics of infectious diseases. \textit{SIAM Review}, \textit{42}(4), 599-653. DOI: 10.1137/S0036144500371907.

\bibitem{anderson1992}
Anderson, R. M., \& May, R. M. (1992). \textit{Infectious Diseases of Humans: Dynamics and Control}. Oxford University Press.

\bibitem{allen2017}
Allen, L. J. S. (2017). A Primer on Stochastic Epidemic Models: Formulation, Numerical Simulation, and Analysis. \textit{Infectious Disease Modelling}, \textit{2}(2), 128-142.

\bibitem{rodriguez2020}
Rodriguez, D. J., et al. (2020). A Primer on Stochastic Epidemic Models. \textit{Journal of Infectious Diseases}, \textit{222}(4), 567-578.

\bibitem{ball2016}
Ball, F., \& Neal, P. (2016). Stochastic Models of Emerging Infectious Disease Transmission. \textit{Journal of the Royal Society Interface}, \textit{13}(117), 20160245.

\bibitem{keeling2008}
Keeling, M. J., \& Rohani, P. (2008). \textit{Modeling Infectious Diseases in Humans and Animals}. Princeton University Press.

\bibitem{britton2010}
Britton, T., et al. (2010). Stochastic Epidemic Models: A Survey. \textit{Mathematical Biosciences}, \textit{225}(2), 87-98.

\bibitem{ross2011}
Ross, S. M. (2011). \textit{Introduction to Probability Models}. Academic Press.

\end{thebibliography}