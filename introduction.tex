\chapter{Introduction}
Infectious diseases necessitate robust models to simulate their spread. Network analysis serves as a powerful tool in this domain, enabling researchers to visualize and predict how diseases propagate through communities. This project utilizes grid-based and network-based simulations to explore the dynamics of disease spread, employing various structures to mirror the complexity of real-world interactions.

\section{Main Challenges}
One of the main challenges in this research is the identification of the source of an infection within a network. This task is particularly daunting due to the diverse and complex nature of network structures, which can significantly affect the propagation paths of infectious agents.

\section{Limitations}
Existing models like the SIR and SIS, while foundational, often fall short in capturing phenomena such as waning immunity and recurrent infections, which are critical for understanding diseases like COVID-19.

\section{Introduction to the SIRS Model}
The SIRS model addresses these shortcomings by incorporating waning immunity, making it particularly relevant for diseases that do not confer lifelong immunity. This model forms the cornerstone of our study.

\begin{figure}[H]
    \centering
    \includegraphics[width=0.8\textwidth]{SIRS_Model_Dynamics.png}
    \caption{SIRS model dynamics}
    \label{fig:SIRS_Model_Dynamics}
\end{figure}

\section{Approach}
Our approach integrates simulating an infectious disease on both grid-based and graph-based models, then applying our source recovery detection algorithms. We will employ various centrality measures to enhance the accuracy of source identification. The goal is to have a clear success rate of infection source recovery over time and to understand the underlying reasons for these results. This allows for a comprehensive analysis of infection dynamics, accommodating the distinct features of each network type. The integration of multiple network models and centrality measures is essential for developing a nuanced understanding of disease spread.

\section{Thesis Statement}
Our comparative study shows a marked improvement in source detection within both fixed and non-fixed boundary conditions for a grid-based model and how these source detection algorithms work on other network models. The core contribution of this thesis lies in its development of a simulation framework that adapts to various network structures, enhancing the accuracy of source detection. This advancement not only improves our theoretical understanding of network-based disease spread but also has practical implications for designing more effective containment measures.