\chapter{Introduction}
\section{Background Information about the Problem}
Infectious diseases have long posed significant threats to public health, necessitating the development of robust models to understand and predict their spread. Network analysis is an invaluable tool in this domain, allowing researchers to visualize and simulate how diseases propagate through different population structures. Traditional models such as Susceptible-Infected-Recovered (SIR) and Susceptible-Infected-Susceptible (SIS) have been fundamental in epidemiological studies. However, these models often fall short in capturing the complexities of real-world disease transmission, particularly when it comes to phenomena like waning immunity and recurrent infections, which are critical for understanding certain diseases such as COVID-19. As human interactions become increasingly complex, there is a pressing need to explore more sophisticated network models that can accurately represent the intricacies of disease spread in various contexts.

\section{Main Challenges}
One of the foremost challenges in this research is accurately identifying the source of an infection within a network. The diverse and intricate nature of network structures can significantly influence the propagation paths of infectious agents, making this task particularly complex. Grid-based models, while straightforward and easy to implement, often do not capture the heterogeneity of real-world interactions. Conversely, complex network models, although more realistic, pose significant computational challenges and require sophisticated algorithms for effective simulation. Balancing these two approaches and ensuring that the models accurately reflect real-world conditions are significant hurdles that this research seeks to overcome.

\section{Thesis Statement}
This project aims to investigate the spread of infectious diseases across various network structures, with a specific focus on both grid-based and graph-based models. The primary aim is to understand how these models can be used to simulate and analyze disease dynamics.

\subsection{Introduction to the SIRS Model}
The Susceptible-Infected-Recovered-Susceptible (SIRS) model, which accounts for waning immunity, is particularly applicable to diseases where immunity is not lifelong. This model rectifies the limitations of conventional epidemiological models by integrating waning immunity. This model forms the cornerstone of our study, enabling a more accurate representation of diseases where individuals can become susceptible again after a period of being recovered.

\subsection{Approach Description}
Our approach integrates simulating the spread of infectious diseases on both grid-based and graph-based models, followed by the application of our source recovery detection algorithms. We employ various centrality measures to enhance the accuracy of source identification. The goal is to achieve a clear success rate of infection source recovery over time and to understand the underlying reasons for these results. This comprehensive analysis accommodates the distinct features of each network type. The integration of multiple network models and centrality measures is essential for developing a nuanced understanding of disease spread.

The main idea of this project lies in its development of a simulation framework that adapts to various network structures, enhancing the accuracy of infection source detection. This project not only improves our theoretical understanding of network-based disease spread but also proves to be useful for assimilating all kind of source detection algorithms.

\section{Limitations}
While this study advances the understanding and implementation of network-based modeling for infectious disease spread, several limitations must be acknowledged. The algorithms employed, especially those involving centrality measures, can be computationally intensive, posing challenges when scaling up to larger networks or running extensive simulations over multiple iterations. The grid-based and network models used in this study rely on certain assumptions, such as homogeneous mixing within subpopulations and static network structures. These assumptions may not fully capture the complexities of real-world scenarios, where networks can be dynamic and heterogeneous. Additionally, the accuracy of the models heavily depends on the parameters chosen, such as infection rates ($\beta$), recovery rates ($\gamma$), and re-susceptibility rates ($\delta$). Inaccurate estimation of these parameters can lead to significant deviations from real-world outcomes.

Boundary effects in grid-based models can significantly influence the accuracy of source detection algorithms. While wrap-around boundary conditions help mitigate some of these effects, they do not entirely eliminate the potential inaccuracies near the edges of the grid. The performance of the models, especially those employing machine learning techniques, depends on the availability and quality of historical data. In many real-world scenarios, such data may be incomplete or noisy, affecting the robustness and generalizability of the models. Although the simulation framework developed in this study shows promise, its scalability to large real-world networks and diverse epidemiological contexts remains an area for further exploration and refinement. The transition from theoretical models to practical applications involves challenges such as handling incomplete data and integrating with real-time surveillance systems. Addressing these limitations in future work will be crucial for enhancing the accuracy, and applicability of network-based models in epidemiological research.
